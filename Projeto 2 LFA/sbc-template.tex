\documentclass[12pt]{article}

\usepackage{sbc-template}
\usepackage{graphicx,url}
\usepackage{listings}
\usepackage{longtable}

\usepackage[brazil]{babel}   
%\usepackage[latin1]{inputenc}  
\usepackage[utf8]{inputenc}  
% UTF-8 encoding is recommended by ShareLaTex

     
\sloppy

\title{Construção de aplicação para geração dos conjuntos FIRST e FOLLOW a partir de uma GLC em linguagem C++}

\author{Acácia dos Campos da Terra\inst{1}, Gabriel Batista Galli\inst{1},\\ João Pedro Winckler Bernardi\inst{1}, Vladimir Belinski\inst{1} }

\address{Ciência da Computação -- Universidade Federal da Fronteira Sul
  (UFFS)\\
  Caixa Postal 181 -- 89.802-112 -- Chapecó -- SC -- Brasil
  \email{\{terra.acacia, g7.galli96, winckler.joao, vlbelinski\}@gmail.com}
}

\begin{document} 

\maketitle

\begin{abstract}
 This paper describes the implementation of an application in C++ programming language for [...] The particularities and operation of the application, such as its planning and code will be commented and analyzed in the work. At the end it will be checked the operation of the application by conducting tests.
\end{abstract}
     
\begin{resumo} 
 O presente trabalho descreve a implementação de uma aplicação em linguagem de programação C++ para [...]. As particularidades e funcionamento da aplicação, tal como seu planejamento e código serão comentados e analisados no trabalho. No final será verificado o funcionamento da aplicação através da realização de testes.

\end{resumo}


\section{Introdução}

O presente trabalho objetiva descrever a construção de uma aplicação em linguagem de programação C++ para [...].

Na Seção \ref{2} será realizada uma breve descrição do que são os conjuntos FIRST  e FOLLOW e suas motivações. Por sua vez, na Seção \ref{3} será realizada uma descrição do que consiste a aplicação. Em sequência, a implementação e planejmento da aplicação serão demonstrados na Seção \ref{4} e os testes que verificam seu funcionamento na Seção \ref{5}. Por fim, na Seção \ref{6} poderão ser encontradas as conclusões acerca do trabalho.

\section{Conjuntos FIRST e FOLLOW}
\label{2}

Será que tem alguma coisa em \cite{menezes:00}?

\section{Descrição da Aplicação}
\label{3}

Objetivo:
Construir aplicação para gerar o conjunto FIRST e o conjunto FOLLOW a partir de uma GLC a ser lida de arquivo de entrada.

Descrição:
A aplicação faz a carga de uma GLC (Gramárica Livre de Contexto) a partir de um arquivo fonte (texto).

Usar notação BNF para as GLCs. Descrever símbolo que representa eps.

Feita a carga, a aplicação gera o conjunto FIRST e em seguida o conjunto FOLLOW da GLC e salva em arquivo de saída.

A GLC pode usar um único símbolo para representar cada token ou pode usar o mnemônico do token nas produções. No segundo caso, a aplicação deve tratar cada mnemônico como um único token.



\section{Planejamento e Implementação}
\label{4}
//MUDAR OS NOMES DOS ARQUIVOS PRA FAZER SENTIDO?!


A implementação da aplicação pode ser encontrada em três arquivos: \texttt{ndfa.cpp}, \texttt{automata.h} e \texttt{automata.cpp}.

No arquivo \texttt{ndfa.cpp} se encontra a função \texttt{main}, onde é realizada a leitura do arquivo de entrada, a impressão de mensagens no prompt a fim de permitir ao usuário o acompanhamento da aplicação e realizadas chamadas às funções responsáveis pelas ações executadas pela aplicação.

Por sua vez, no arquivo \texttt{automata.h} podem ser encontradas as definições das constantes utilizadas no trabalho, das estruturas de dados e os protótipos das funções implementadas em \texttt{automata.cpp}. Cabe destacar que a \texttt{struct symbol} representa... , enquanto a \texttt{struct transition}...

Em \texttt{automata.cpp} têm-se  $7$ funções, as quais serão explicadas uma a uma a seguir.

--DESTACAR PARÂMETROS, FUNCIONALIDADE, ESTRUTURAS UTILIZADAS, RETORNO/SAÍDA DE CADA UMA DS FUNÇÕES ABAIXO:

Inicialmente, em relação à função \texttt{readgrammar}...

Por sua vez, a função \texttt{first}...

Em relação à \texttt{printfirst} têm-se que...

No que lhe diz respeito, a função \texttt{follow}...

Para.. foi implementada a função \texttt{printfllw}, que...

A respeito da função \texttt{printfa} têm-se que...

Por fim, a função \texttt{csv}...


\section{Testes}
\label{5}

Para a realização dos testes foram criados quatro arquivos, nomeados \texttt{test.in}, \texttt{frst.in}, \texttt{frst2.in} e \texttt{frst3.in}. Cada um desses arquivos armazena uma Gramática Livre de Contexto a partir da qual foram gerados os conjuntos FIRST e FOLLOW relacionados a essa GLC.

A partir da realização dos testes pôde ser verificado que os resultados obtidos condizem com as saídas esperadas, essas que podem ser verificadas em dois arquivos \texttt{.csv} gerados, um para o conjunto FIRST e outro para o conjunto FOLLOW. Isso demonstra que a aplicação construída atende aos objetivos a qual se propõe.

\section{Conclusão}
\label{6}

Texto...

\bibliographystyle{sbc}
\bibliography{sbc-template}

\end{document}